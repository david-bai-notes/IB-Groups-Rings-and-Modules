\section{Alternating Groups}
Conjugation in $S_n$ is relatively easy.
Permutations in $S_n$ are just the cycle-type-preserving maps.
\begin{example}
    In $S_5$, we have
    \begin{center}
        \begin{tabular}{c|c}
            Cycle Type&Number of Elements\\
            \hline
            $1$&$1$\\
            $2^1$&$10$\\
            $2^2$&$15$\\
            $3^1$&$20$\\
            $2^13^1$&$20$\\
            $4^1$&$30$\\
            $5^1$&$24$\\
            \hline
            Total&$120$
        \end{tabular}
    \end{center}
\end{example}
Let $g\in A_n$, then $C_{A_n}(g)=C_{S_n}(g)\cap A_n$.
So if there is an odd permutation that commutes with $g$, then
$$|C_{A_n}(g)|=\frac{1}{2}|C_{S_n}(g)|,|\operatorname{ccl}_{A_n}(g)|=|\operatorname{ccl}_{S_n}(g)|$$
Otherwise,
$$|C_{A_n}(g)|=|C_{S_n}(g)|,|\operatorname{ccl}_{A_n}(g)|=\frac{1}{2}|\operatorname{ccl}_{S_n}(g)|$$
\begin{example}
    Consider $(12)(34)\in A_5$.
    It commutes with $(12)$, so the conjugacy class stays the same.\\
    Similar for $(123)$ which commutes with $(45)$.\\
    But if we consider $(12345)$, things go different.
    Note that if $h$ is in its centralizer, then $h(12345)h^{-1}=(h(1)h(2)h(3)h(4)h(5))$, so $h$ has to be some power of $(12345)$, in particular $h$ is even.
    Therefore in this case the conjugacy class splits into two.\\
    So the conjugacy classes of $A_5$ have sizes $1,15,20,12,12$, so by the same trick we used, $A_5$ is simple.
\end{example}
\begin{lemma}
    $A_n$ is generated by $3$-cycles.
\end{lemma}
\begin{proof}
    Each $\sigma\in A_n$ is a product of an even number of transpositions.
    So it suffices to prove that the product of any two transpositions is a product of $3$-cycles.\\
    For $a,b,c,d$ different, we have
    $$(ab)(bc)=(abc),(ab)(cd)=(acb)(acd)$$
    As desired.
\end{proof}
\begin{lemma}
    If $n\ge 5$, then all $3$-cycles in $A_n$ are conjugate.
\end{lemma}
\begin{proof}
    We claim that any $3$-cycle is conjugate to $(123)$.
    Note that for $a,b,c$ different, $(abc)=\sigma(123)\sigma^{-1}$.
    If $\sigma$ is odd, then we still have $(abc)=\pi(123)\pi^{-1}$ where $\pi=\sigma(45)$ and $\pi$ is even.
\end{proof}
\begin{theorem}
    $A_n$ is simple for $n\ge 5$.
\end{theorem}
\begin{proof}
    Let $1\neq N\unlhd A_n$,.
    It suffices to show that $N$ contains a $3$-cycle, then by the preceding lemmas, $N=A_n$.
    We start by choosing a non-trivial element $1\neq\sigma\in N$ and write it as a product of disjoint cycles.\\
    Case 1: $\sigma$ contains an $r$-cycle where $r\ge 4$.
    WLOG $\sigma=(12\ldots r)\tau$ where $\tau$ fixes $1,2,\ldots,r$.
    And let $\delta=(123)$, we compute
    $$N\ni\sigma^{-1}\delta^{-1}\sigma\delta=(r\ldots 21)(132)(12\ldots r)(123)=(23r)$$
    So $N=A_n$.\\
    Case 2: $\sigma$ contains two $3$-cycles.
    WLOG $\sigma=(123)(456)\tau$ where $\tau$ fixes $1,2,\ldots,6$.
    We let $\delta=(124)$, then $N\ni\sigma^{-1}\delta^{-1}\sigma\delta=(12436)$ which is a $5$-cycle.
    By Case 1, $N$ also contains a $3$-cycle.\\
    Case 3: $\sigma$ contains two $2$-cycles, where we set WLOG $\sigma=(12)(34)\tau$ where $\tau$ fixes $1,2,3,4$.
    We let $\delta=(123)$, so $N\ni\sigma^{-1}\delta^{-1}\sigma\delta=(14)(23)=\pi$.
    Then let $\epsilon=(235)$ (note that we used $n\ge 5$ here), then $N\ni\pi^{-1}\epsilon^{-1}\pi\epsilon=(253)$, which is a $3$-cycle.\\
    It remains to consider cycle types $2^1,3^1,2^13^1$.
    But $2^1,2^13^1$ are not in $A_n$ since they are not even, and for $3^1$ it follows immediately.
\end{proof}

