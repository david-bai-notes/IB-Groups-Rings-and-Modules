\section{Introduction}
In this course, we will continue the analysis of groups from last year, to topics like simple groups, $p$-(sub)groups.
The main highlight in this part will be the Sylow's Theorems.\\
Second, we will look into another algebraic structure, rings, where you have two operations linked together by certain axioms (like you can add, substract and multiply).
Examples include $\mathbb Z$, $\mathbb C[x]$ and so on.
More important examples are ``rings of integers'' (rings behave like integers), like $\mathbb Z[i]$ or $\mathbb Z[\sqrt{2}]$.
The unique (or not unique) factorizations in these rings give rise to methods in number theories.
Another important thing is the ring of polynomials, which is the core object studies in Algebraic Geometry.\\
Fields, where divisions are possible for nonzero elements and multiplications commute, are also studied.
Examples are $\mathbb Q,\mathbb R,\mathbb C,\mathbb Z/p\mathbb Z$.\\
Lastly, we study modules, which are like vector spaces where one replace the underlying field by a ring.
In particular, every vector space is a module.
We will classify finitely generated modules over certain rings, which allows us to prove Jordan Normal Form for modules and to classify finite abelian groups.