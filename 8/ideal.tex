\section{Ideals, Quotients, and Isomorphism Theorems}
\subsection{Definitions}
\begin{definition}
    Let $R,S$ be rings.
    A function $\phi:R\to S$ is called a ring homomorphism if it for any $r_1,r_2\in R$\\
    1. $\phi(r_1+r_2)=\phi(r_1)+\phi(r_2)$.\\
    2. $\phi(r_1r_2)=\phi(r_1)\phi(r_2)$.\\
    3. $\phi(1_R)=1_S$.\\
    If additionally this map is bijective, we call this an isomorphism.\\
    The kernel of $\phi$ is the set $\ker\phi=\{r\in R:\phi(r)=0_S\}$.
\end{definition}
\begin{lemma}
    A ring homomorphism $\phi:R\to S$ is injective iff $\ker\phi=\{0_R\}$. 
\end{lemma}
\begin{proof}
    $\phi$ is also a group homomorphism $(R,+)\to (S,+)$.
\end{proof}
\begin{definition}
    A subset $I\subset R$ is called an ideal, written as $I\unlhd R$, if $(I,+)\le (R,+)$ and $\forall r\in R,rI\subset I$.
\end{definition}
\begin{remark}
    Note that an ideal needs not to be a subring since it might not contain $1$.
    In fact, if $1\in I$, then $\forall r\in R, r=r1\in I$, so $I=R$.
    In general, if $I$ contains a unit $u$, then $\forall r\in R,r=(ru^{-1})u\in I$, so again $I=R$.\\
    Therefore, a field can only have two ideals, $\{0\}$ and itself.
\end{remark}
An ideal $I\neq R,\{0\}$ is called proper.
\begin{lemma}
    Let $\phi:R\to S$ be a ring homomorphism, then $\ker\phi\unlhd R$.
\end{lemma}
\begin{proof}
    $\ker\phi$ is obviously an additive subgroup of $R$.
    Also $\forall r\in R,i\in I,\phi(ri)=\phi(r)\phi(i)=\phi(r)0_S=0_S\implies ri\in\ker\phi$.
    Hence $\ker\phi\unlhd R$.
\end{proof}
\begin{lemma}
    The only ideals of $\mathbb Z$ are $n\mathbb Z,n\in\mathbb Z$.
\end{lemma}
\begin{proof}
    All of them are ideals and these are all possible additive subgroups.
\end{proof}
\begin{definition}
    For $a\in R$, the ideal generated by $a$ is the ideal $(a)=\{ra:r\in R\}$.
\end{definition}
Note that $(a)$ is the smallest ideal that contains $a$.
More generally,
\begin{definition}
    For $a_1,\ldots,a_n\in R$, the ideal generated by them is the ideal $(a_1,\ldots,a_n)=\{r_1a_1+\cdots+r_na_n:r_i\in R\}$.
\end{definition}
\begin{definition}
    An ideal $I\unlhd R$ is principle if $I=(a)$ for some $a\in R$.
\end{definition}
So every ideal of $\mathbb Z$ is principle.
\begin{theorem}
    Let $I$ be an ideal of $R$, then we take the quotient $R/I$ as if they are additive groups.
    We can define the multiplication on $R/I$ by defining $(a+I)(b+I)=ab+I$ which is well-defined and makes $R/I$ a ring.
\end{theorem}
Such $R/I$ is called the quotient ring.
Note that the canonical projection (or quotient map) $\pi:R\to R/I$ becomes a ring homomorphism with kernel $I$, hence every ideal is the kernel of a homomorphism.
\begin{proof}
    If $a+I=a'+I,b+I=b'+I$, then $a-a',b-b'\in I$, so $ab-a'b'=a(b-b')+(a-a')b'\in I$, therefore $ab+I=a'b'+I$, hence this multiplication is well-defined.
    The rest follows.
\end{proof}
\begin{example}
    1. For $R=\mathbb Z$, then the quotients are $\mathbb Z/n\mathbb Z$ with modulo $n$ addition and multiplication.\\
    2. Consider the ideal generated by $X$ inside the poynomial $R[X]$, then $(X)$ consists of polynomials without a constant term.
    The quotient is then $R[X]/(X)\cong R$ with the isomorphism $r(X)\mapsto r$.\\
    3. Consider the ring of real polynomials $\mathbb R[X]$ and the ideal $(X^2+1)$, then $\mathbb R[X]/(X^2+1)=\{f(X)+(X^2+1):f(X)\in R[X]\}$.
    Now $\mathbb R$ is a field, so every nonzero element is a unit, hence we can always do division algorithm.
    So by applying this algorithm on $f(X)$, we have $\mathbb R[X]/(X^2+1)=\{a+bX+(X^2+1):f(X)\in R[X]\}$.
    Now suppose $a+bX+(X^2+1)=a'+b'X+(X^2+1)$, then $(b-b')X+(a-a')=Q(X)(X^2+1)$ for some polynomial $Q$, but by looking at the degree, we must have $Q=0$, therefore $a=a',b=b'$, so each cosets are uniquely represented like this.
    We then identify $a+bX+(X^2+1)\mapsto a+bi\in\mathbb C$, but this is a ring isomorphism, so $\mathbb R[X]/(X^2+1)\cong\mathbb C$.
\end{example}
\subsection{Isomorphism Theorems of Rings}
\begin{theorem}[(First) Isomorphism Theorem]\label{ring_iso}
    Let $\phi:R\to S$ be a ring homomorphism, then $\ker\phi$ is an ideal of $R$ and the quotient ring $R/\ker\phi$ is isomorphic to $\operatorname{Im}\phi\le S$.
\end{theorem}
\begin{proof}
    We already saw that the kernel is an ideal and the quotient as additive group is isomorphic to $\operatorname{Im}\phi$ which we know is a subgroup of $(S,+)$.
    Also $\operatorname{Im}\phi$ is closed under multiplication since $\phi(r)\phi(s)=\phi(rs)$.
    In addition $\phi(1_R)=1_S$, so it is a subring of $S$.
    Consider $\Phi:R/\ker\phi\to\operatorname{Im}\phi$ which takes a coset $r+\ker\phi$ to $\phi(r)$.
    This is well defined from results in groups.
    It is obviously also a bijection and a group homomorphism under addition.
    To check it is a ring homomorphism, we have $\Phi(1_R+\ker\phi)=\phi(1_R)=1_S$.
    Also, $\Phi((r+\ker\phi)(s+\ker\phi))=\Phi(rs+\ker\phi)=\phi(rs)=\phi(r)\phi(s)=\Phi(r+\ker\phi)\Phi(s+\ker\phi)$.
    So it is a ring isomorphism.
\end{proof}
\begin{corollary}[Second Isomorphism Theorem]
    Let $R\le S$ and $J\unlhd S$, then $R\cap I\unlhd R$ and $R+J\le S$ and
    $$R/(R\cap J)\cong (R+J)/J\le S/J$$
\end{corollary}
\begin{proof}
    $R\cap I\unlhd R$ and $R+J\le S$ are trivial.
    Now consider the map $\phi:R\to S/J$ by $\phi(r)=r+J$.
    It is obviously a well-defined ring homomorphism as the composition of the inclusion $R\to S$ and the quotient map $S\to S/J$.
    Its image is $(S+J)/J$ and its kernel is $R\cap J$.
    The result follows from Theorem \ref{ring_iso}.
\end{proof}
Analogous to the situation in groups, we have (or want to have) a bijection between certain ideals of the ring $R$ and the ideals of the quotient ring $R/I$.
Start with an ideal $K$ containing $I$, we can send it to $\{r\in R:r+I\in K\}$.
Its inverse is just $J\mapsto J/I$.
This motivates the Third Isomorphism Theorem.
\begin{corollary}[Third Isomorphism Theorem]
    Let $I,J\unlhd R$ such that $I\subset J$, then $J/I\unlhd R/I$ and
    $$(R/I)/(J/I)\cong R/J$$
\end{corollary}
\begin{proof}
    Consider $\phi:R/I\to R/J$ by $r+I\mapsto r+J$.
    Since $I\subset J$, this is well-defined and obviously a ring homomorphism with kernel $J/I$.
    Finish by Theorem \ref{ring_iso}.
\end{proof}
\begin{example}
    There is a surjective ring homomorphism $\mathbb R[X]\to\mathbb C$ by
    $$\sum_{k=0}^na_kX^k\mapsto \sum_{k=0}^na_ki^k$$
    Then the kernel, by division algorithm, would be $(X^2+1)$, hence we immediately obtain $\mathbb R[X]/(X^2+1)\cong\mathbb C$ by Thoerem \ref{ring_iso}
\end{example}
\begin{example}[Characteristic of a Ring]
    For a ring $R$, there is an unique ring homomorphism $\mathbb Z\to R$.
    The uniqueness is obvious since a ring homomorphism must map the multiplicative identity to multiplicative identity.
    The existence can be shown by simply constructing $\iota:n\mapsto 1_R+1_R+\cdots +1_R$ where there are $n$ of $1_R$'s added together.
    Similarly $\iota:-n\mapsto -(1_R+1_R+\cdots +1_R)$ where again there are $n$ of $1_R$'s in the bracket.
    So $\ker\iota\unlhd\mathbb Z$, hence $\ker\iota=n\mathbb Z$ for some $n\in\mathbb N_0$.
\end{example}
\begin{definition}
    We say $n$ is the characteristic $\operatorname{char}(R)$ of $R$.
\end{definition}
By Theorem \ref{ring_iso}, $\mathbb Z/n\mathbb Z\cong\operatorname{Im}\iota\le R$.
\begin{example}
    $\mathbb Z,\mathbb Q,\mathbb R,\mathbb C$ all have characteristic $0$, and $\mathbb Z/p\mathbb Z$ has characteristic $p$.
\end{example}