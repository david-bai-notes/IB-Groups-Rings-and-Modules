\section{Groups}
\subsection{Basics}
\begin{definition}
    A group is a pair $(G,\cdot)$ consisting of a set $G$ and a function $\cdot:G\times G\to G$ such that\\
    1. $\forall a,b,c\in G,(a\cdot b)\cdot c=a\cdot(b\cdot c)$.\\
    2. $\exists e\in G,\forall g\in G,e\cdot g=g\cdot e=e$.\\
    3. $\forall g\in G,\exists g^{-1}\in G,g\cdot g^{-1}=g^{-1}\cdot g=e$.
\end{definition}
\begin{remark}
    1. To check something is a group, we need also to check that $\cdot$ is well-defined.
    This is sometimes called the closure axiom, i.e. $\forall a,b\in G, a\cdot b\in G$.\\
    2. If using additive notation, we write $0$ for $e$ and if we are using mutiplicative notation we write $1$ for $e$.
\end{remark}
\begin{definition}
    For a group $(G,\cdot)$, a subset $H\subset G$ is a subgroup if $(H,\cdot|_{H\times H})$ is a well-defined group.
    In this case, we write $H\le G$.
\end{definition}
\begin{remark}
    A nonempty subset $H\subset G$ is a subgroup iff $\forall a,b\in H,ab^{-1}\in H$.
\end{remark}
\begin{example}
    1. $(\mathbb Z,+)\le(\mathbb Q,+)\le(\mathbb R,+)\le\cdots$.\\
    2. The cyclic group $C_n$ of order $n$ and the dihedral group $D_{2n}$ of isometries preserving a regular $n$-gon.\\
    3. Symmetric groups $S_n$ of the permutations of $n$ letters and alternatig groups $A_n$ containing all even permutations of those $n$ letters.\\
    4. The quarterion group $Q_8=\{\pm 1,\pm i,\pm j,\pm k\}$.\\
    5. The group $\operatorname{GL}_n(\mathbb F)$ of all invertible matrices over a field $\mathbb F$ is a group.
    We can also get $\operatorname{SL}_n(\mathbb F)$ consisting of those with determinant $1$.
\end{example}
\begin{definition}
    The direct product of groups $G,H$ is a group $G\times H$ under the operation
    $$\forall g,g'\in G,h,h'\in H,(g,h)(g',h')=(gg',hh')$$ 
\end{definition}
For a subgroup $H$ of $G$, the left cosets of $H$ are the sets of the form $gH,g\in G$.
It is obvious that cosets partition $G$ and all of them have the same cardinality.
So immediately,
\begin{theorem}[Lagrange's Theorem]
    If $G$ is finite and $H\le G$, then $|H|$ divides $|G|$.
\end{theorem}
\begin{definition}
    The value $|G|/|H|$ is called the index $|G:H|$ of $H$ in $G$.
\end{definition}
There is a partial converse to this theorem which we shall introduce soon.
\begin{theorem}[First Sylow's Theorem]
    If $|G|=p^nm$ where $p$ is a prime and $p\nmid m$, there is a subgroup $H\le G$ such that $|H|=p^n$
\end{theorem}
From which Cauchy's Theorem is immediate.
\begin{definition}
    For $g\in G$, the least $n$ such that $g^n=1$ is called the order of $g$.
    If there is no such integer, we say $g$ has infinite order.
\end{definition}
\begin{remark}
    1. If $g$ has order $d$, then $g^n=1\iff d|n$.\\
    2. $n||G|$ by considering the subgroup $\langle g\rangle$ generated by $g$.
\end{remark}
\begin{definition}
    A subgroup $H\le G$ is called a normal subgroup of $G$, written as $H\unlhd G$, if $\forall g\in G,gHg^{-1}=H$.
\end{definition}
\begin{proposition}
    If $H\unlhd G$, then the set of left cosets $gH$ is a group, called the quotient group $G/H$, under the operation $(aH)(bH)=abH$.
\end{proposition}
\begin{proof}
    Trivial to check that the operation is well-defined.
    The rest is obvious.
\end{proof}
\subsection{The Isomorphism Theorems}
\begin{definition}
    Let $G,H$ be groups, a map $\phi:G\to H$ is called a group homomorphism if $\forall g,g'\in G,\phi(gg')=\phi(g)\phi(g')$.
    The kernel of $\phi$ is defined as $\ker\phi=\{g\in G:\phi(g)=1\}\le G$.
    The image of $\phi$ is defined as $\operatorname{Im}\phi=\{h\in H:\exists g\in G,\phi(g)=h\}\le H$
\end{definition}
Obviously $\ker\phi\unlhd G$.
The concerse is also true: any normal subgroup of $G$ is the kernel of the canonical projection of $G$ to the quotient.
\begin{definition}
    An isomorphism is a bijective homomorphism.
    We say $G,H$ are isomorphic, or $G\cong H$, if there is an isomorphism between them.
\end{definition}
\begin{proposition}
    If $\phi$ is an isomorphism, so is $\phi^{-1}$.
\end{proposition}
\begin{proof}
    Trivial.
\end{proof}
\begin{theorem}[Isomorphism Theorem]
    Let $\phi:G\to H$ be a group homomorphism, then $G/\ker\phi\cong\operatorname{Im}\phi$.
\end{theorem}
\begin{proof}
    The map $\tilde{\phi}:G/\ker\phi\to\operatorname{Im}\phi$ by $g\ker\phi\mapsto \phi(g)$ is a well-defined isomorphism.
\end{proof}
\begin{example}
    The function $\exp:\mathbb C\to\mathbb C^\star$ is a homomorphism with kernel
    $$\ker\exp=\{z\in C:\exp(z)=1\}=2\pi\mathbb Z,\operatorname{Im}\exp=\mathbb C^\star$$
    Hence $\mathbb C/2\pi\mathbb Z=\mathbb C^\star$.
\end{example}
The Isomorphism Theorem is sometimes called the \textit{First} Isomorphism Theorem.
There are other isomorphism theorems, but they are all corollaries of the first one.
\begin{corollary}[Second Isomorphism Theorem]
    Consider a group $G$ and subgroups $H\le G,K\unlhd G$, then the set $HK=\{hk:h\in H,k\in K\}$ is a subgroup of $G$.
    Also $H\cap K\unlhd H$.
    Then we have $HK/K\cong H/H\cap K$.
\end{corollary}
\begin{proof}
    The (normal) subgroup conditions in $HK,H\cap K$ are trivial.
    The function $\phi:H\to G/K$ by $h\mapsto hK$ is a homomorphism because it is the composition of the inclusion $H\to G$ and the projection $G\to G/K$.
    So immediately $\ker\phi=H\cap K$ and $\operatorname{Im}\phi=HK/K$, and the result follows.
\end{proof}
\begin{remark}
    There is a natural bijection from the subgroups of $G/K$ and the subgroups of $G$ containing $K$ by $X\mapsto \{g\in G:gK\in X\}$.
    How about normal subgroups?
    Turns out we have the Third Isomorphism Theorem to describe this.
\end{remark}
\begin{corollary}[Third Isomorphism Theorem]
    Suppose $H,K$ are normal subgroups of $G$ and $K\subset H$.
    SO $K\unlhd H$ and $H/K\unlhd G/K$ with
    $$G/H\cong (G/K)/(H/K)$$
\end{corollary}
\begin{proof}
    Let $\phi:G/K\to G/H$ by $gK\mapsto gH$ which is clearly a well-defined surjective homomorphism, then $\ker\phi=H/K$.
    The result then follows from the First Isomorphism Theorem.
\end{proof}
\subsection{Simple Groups}
If $K\unlhd G$, then the study of the groups $K$ and $G/K$ gives some information about $G$ (but not all).
But sometimes this approach fails due to the lack of normal proper subgroups.
\begin{definition}
    A group is called simple if it has no normal proper subgroups.
\end{definition}
\begin{lemma}
    An abelian group $G$ is simple if and only if it is isomorphic to the cyclic group of order $p$ for some prime $p$.
\end{lemma}
\begin{proof}
    An abelian group is simple if and only if it has no proper subgroup.
    It is obvious that for any prime $p$, $C_p$ has no proper subgroup due to Lagrange's Theorem, so we proceed to the converse.
    Note that any nonidentity element $x\in G$ must generate $G$, otherwise the subgroup generated by it will be a proper subgroup of $G$.
    However this means that $G$ is cyclic, so immediately to make it simple $G$ must be isomorphic to $C_p$ for some prime $p$.
\end{proof}
\begin{lemma}
    If $G$ is a finite group, then it has a composition series
    $$1=G_0\lhd G_1\lhd\cdots\lhd G_m=G$$
    with $G_{i}/G_{i-1}$ is simple for each $i$.
\end{lemma}
\begin{proof}
    Induction on $|G|$.
    By the Third Isomorphism Theorem a normal proper subgroup $H\lhd G$ with maximal order must have $G/H$ simple, so the result follows.
\end{proof}