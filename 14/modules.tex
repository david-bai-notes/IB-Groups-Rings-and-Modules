\section{Modules}
\subsection{Definition and Examples}
\begin{definition}
    A module over a ring $R$ (an $R$-module) is a triple $(M,+,\cdot)$ where $(M,+,0)$ for some $0\in M$ is an abelian group and $\cdot:R\times M\to M$ (called scalar multiplication) satisfies, for any $r,r_1,r_2\in R,m,m_1,m_2\in M$:\\
    1. $(r_1+r_2)\cdot m=r_1\cdot m+r_2\cdot m$.\\
    2. $r\cdot (m_1+m_2)=r\cdot m_1+r\cdot m_2$.\\
    3. $r_1\cdot(r_2\cdot m)=(r_1r_2)\cdot m$.\\
    4. $1\cdot m=m$.
\end{definition}
The function $\cdot$ is called the scalar multiplication and is omitted from writing sometimes.
\begin{remark}
    To show something is a module, we also need to check closure (that is $+,\cdot$ are well-defined).
\end{remark}
\begin{example}
    1. If $R$ is a field, then an $R$-module $M$ is a vector space over $R$.\\
    2. A $\mathbb Z$-module is precisely the same as an abelian group as the scalar multiplication can be uniquely defined by $n\cdot a=a+\cdots +a$ for $n$ many copies of $a$.\\
    3. Consider the ring $R=\mathbb F[X]$ for a field $\mathbb F$ and $V$ a vector space over $\mathbb F$.
    Consider $\alpha:V\to V$ an endomorphism.
    We can make $V$ an $R$-module over the scalar multiplication $\mathbb F[X]\times V\to V$ by $(f,v)\mapsto f(\alpha)(v)$.
    Note that different choice of $\alpha$ makes $V$ a different module.
    We sometimes write this as $V_\alpha$.
\end{example}
There are some general construction methods to produce a module.
\begin{example}
    1. For any ring $R$, $R^n$ is an $R$-module by $r\cdot(r_1,\ldots,r_n)=(rr_1,\ldots,rr_n)$ for $r,r_i\in R$.
    In particular, when $n=1$, $R$ itself is an $R$-module.\\
    2. If $I$ is an ideal, then $I$ is an $R$-module by $r\cdot i=ri$ for $r\in R,i\in I$.\\
    3. If $I$ is an ideal, then $R/I$ is an $R$-module by $r\cdot(s+I)=rs+I$ for $r,s\in R$.\\
    4. If $\phi:R\to S$ is a ring homomorphism, then any $S$-module $M$ is also an $M$-module by $r\cdot m=\phi(r)\cdot m$ for $r\in R,m\in M$.
    In particular, if $R\le S$, then any $S$-module can be viewed as an $R$-module.
\end{example}
\begin{definition}
    Let $M$ be an $R$-module, a subset $N\subset M$ is called a $R$-submodule of $M$, written as $N\le M$, if $(N,+)\le (M,+)$ and for any $r\in R,n\in N$, we have $r\cdot n\in N$.
\end{definition}
\begin{example}
    1. Any $R$-submodule of $R$ is an ideal.\\
    2. When $R$ is a field, then an $R$-module is a vector space, then a submodule is a vector subspace.
\end{example}
\begin{definition}
    If $N$ is a $R$-submodule of $M$, we can form the quotient $M/N$ by taking the quotient group under addition.
    We can make it as an $R$-module by specifying the scalar multiplication $r\cdot (m+N)=r\cdot m+N$.
\end{definition}
We can check easily that the scalar multiplication defined in this way is well-defined and makes $M/N$ an $R$-module.
\subsection{Homomorphisms}
\begin{definition}
    Let $M,N$ be $R$-modules, then a function $f:M\to N$ is a homomorphism of $R$-modules (or $R$-module map) if $f$ is a homomorphism of groups under addition and $\forall r\in R,m\in M,f(r\cdot m)=r\cdot f(m)$.\\
    A bijective homomorphism is called an isomorphism, and two $R$-modules $M,N$ are called isomorphic (written as $M\cong N$) if there is an isomorphism between them.
\end{definition}
\begin{example}
    When $R$ is a field, a homomorphism of $R$-modules is a linear map.
\end{example}
\begin{theorem}[(First) Isomorphism Theorem for Modules]
    Suppose $M,N$ are $R$-modules and $f:M\to N$ is a homomorphism of $R$-modules, then $\ker f\le M,f(M)\le N$ and $M/\ker f\cong f(M)$.
\end{theorem}
\begin{proof}
    Similar to before.
\end{proof}
\begin{theorem}[Second Isomorphism Theorem]
    Let $A,B$ be $R$-submodules of an $R$=module $M$, then $A+B=\{a+b:a\in A,b\in B\}\le M$ and $A\cap B\le M$.
    Moreover, $A/(A\cap B)\cong (A+B)/B$.
\end{theorem}
\begin{proof}
    Use the First Isomorphism Theorem.
\end{proof}
To motivate the Third Isomorphism Theorem, we note that for $R$-modules $N\le M$, we have the correspondance between the submodules of $M/N$ and the submodules of $M$ containing $N$.
\begin{theorem}[Third Isomorphism Theorem]
    Suppose $N\le L\le M$ are $R$-modules, then $M/L\cong (M/N)/(L/N)$.
\end{theorem}
\begin{proof}
    Same.
\end{proof}
In partricular, these are all true for vector spaces by taking the ring to be a field.
One can compare these results to familiar results in linear algebra (e.g. the First Isomorphism Theorem implies the Rank-Nullity Theorem).
\subsection{Finitely Generated Modules}
\begin{definition}
    Let $M$ be an $R$-module, and $m\in M$, then the submodule $Rm$ generated by $m$ is the smallest $R$-submodule of $M$ containing $m$, i.e. $Rm=\{r\cdot m:r\in R\}$.
\end{definition}
\begin{definition}
    Let $M$ be an $R$-module.
    $M$ is called cyclic if $M=Rm$ for some $m\in M$.
    $M$ is finitely generated if $\exists m_1,\ldots,m_n\in M$ such that $Rm_1+\cdots Rm_n=M$.
\end{definition}
\begin{lemma}
    An $R$-module $M$ is cyclic iff $M$ is isomorphic as an $R$-module to $R/I$ for some $I\unlhd R$.
\end{lemma}
\begin{proof}
    If $M$ is cyclic, write $M=Rm$, then there is a surjective $R$-module homomorphism $R\to M$ by $r\mapsto r\cdot m$ so the claim follows by the First Isomorphism Theorem.\\
    Conversely If $M\cong R/I$, then $M\cong R/I=R(1+I)$.
\end{proof}
\begin{lemma}
    An $R$-module $M$ is finitely generated iff there exists a surjective $R$-module homomorphism from $f:R^n\to M$ for some $n$.
\end{lemma}
\begin{proof}
    If $M$ is finitely generated, then $M=Rm_1+\cdots +Rm_n$ where $m_i\in M$, so we can take $f(r_1,\ldots,r_n)=r_1m_1+\cdots +r_nm_n$.\\
    Conversely, if such a map $f$ exists, then $M=Rf(e_1)+\cdots+Rf(e_n)$, then $e_i$ has $1$ in $i^{th}$ entry and $0$ in $j^{th}$ entry for any $j\neq i$.
\end{proof}
\begin{corollary}
    The quotient of a finitely generated $R$-module is a finitely generated $R$-module.
\end{corollary}
\begin{proof}
    Obvious from the preceding lemma.
\end{proof}
\begin{remark}
    A submodule of a finitely generated $R$-module needs not be finitely generated.
    For example, we can take a non-Noetherian ring $R$ itself as an $R$-module and consider a non-finitely generated ideal of it.
\end{remark}
\begin{lemma}
    Let $R$ be an integral domain, then every $R$-submodule of a cyclic $R$-module is cyclic iff $R$ is a PID.
\end{lemma}
\begin{proof}
    $R$ itself is a cyclic $R$-module, so if all $R$-submodules of it are cyclic, then all of its ideals are generated by one element, so $R$ is a PID.\\
    Conversely, if $R$ is a PID and $M$ is a cyclic $R$-module, so $M\cong R/I$ for $I\unlhd R$, so the $R$-submodules of $M$ are in the form $J/I$ for $I\subset J\unlhd R$.
    Now since $R$ is a PID, $J$ is principal, so $J/I$ is cyclic.
\end{proof}
\begin{theorem}
    Let $R$ be a PID, and $M$ an $R$-module.
    Suppose $M$ is generated by $n$ elements, then any $R$-submodule $N$ of $M$ can also be generated by at most $n$ elements.
\end{theorem}
\begin{proof}
    $n=1$ is the preceding lemma.
    For general $n$, we proceed by induction.
    Suppose $M=Rx_1+\cdots Rx_n$.
    Let $M_i=Rx_1+\cdots Rx_i$ and $0=M_0\le M_1\le\cdots\le M_n=M$.
    So we have
    $$0=M_0\cap N\le M_1\cap N\le\cdots\le M_n\cap N=N$$
    Then the $R$-module map $M_i\cap N\to M_i/M_{i-1}$ by $m\mapsto m+M_{i-1}$ has kernel $M_{i-1}\cap N$.
    Hence
    $$(M_i\cap N)/(M_{i-1}\cap N)\cong M'\le M_i/M_{i-1}$$
    But $M_i/M_{i-1}$ is cyclic by hypothesis, so by preceding lemma, $(M_i\cap N)/(M_{i-1}\cap N)$ is also cyclic and is generated by $y_i+M_{i-1}\cap N$ where $y_i\in M_i\cap N$.
    Therefore $M_i=M_{i-1}\cap N+Ry_i$.
    It follows that $M_i\cap N=Ry_1+\cdots +Ry_i$.
    In particular, $N=M_n\cap N=Ry_1+\cdots+Ry_n$, so $N$ is generated by $n$ elements.
\end{proof}
\begin{example}
    Take $R=\mathbb Z$, then we know that any subgroup of $\mathbb Z^n$ can be generated by $n$ elements.
\end{example}