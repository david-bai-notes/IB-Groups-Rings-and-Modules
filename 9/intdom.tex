\section{Integral Domains, Maximal and Prime Ideals}
\subsection{Integral Domains}
\begin{definition}
    An integral domain is a ring $R$ with $0\neq 1$ and $ab=0$ implies $a=0$ or $b=0$.
\end{definition}
\begin{definition}
    In a ring $R$, an element $a\neq 0$ is called a zero divisor if $\exists b\in R,b\neq 0,ab=0$.
\end{definition}
So an integral domain is a ring without zero divisors.
\begin{example}
    1. All fields are integral domains.\\
    2. Any subring of an integral domain is an integral domain.
    Hence $\mathbb Z[i]\le\mathbb C$ is an integral domain.\\
    3. (non-example) $\mathbb Z\times\mathbb Z$ is not an integral domain since $(1,0)(0,1)=(0,0)$.
\end{example}
\begin{lemma}
    If $R$ is an integral domain, so is $R[X]$.
\end{lemma}
\begin{proof}
    Let $f,g\in R[X]$ be nonzero polynomials.
    Suffice to show that $\deg(fg)=\deg(f)+\deg(g)$.
    Indeed, if
    $$f(X)=\sum_{k=0}^na_kX^k,g(X)=\sum_{k=0}^mb_kX^k,a_n,b_m\neq 0$$
    Then $f(X)g(X)=a_nb_mX^{n+m}+\cdots$, but since $R$ is an integral domain, $a_nb_m\neq 0$, therefore $\deg(fg)=n+m=\deg(f)+\deg(g)$.
\end{proof}
\begin{lemma}
    Let $R$ be an integral domain and $0\neq f\in R[X]$.
    Then the number of roots of $f$ in $R$ is at most $n$.
\end{lemma}
\begin{proof}
    Exercise.
\end{proof}
\begin{theorem}
    Any finite subgroup of the multiplicative group of a field is cyclic.
\end{theorem}
\begin{example}
    $(\mathbb Z/p\mathbb Z)^\times$ is cyclic.\\
    Also, $U_m=\{x^m=1:x\in\mathbb C\}$ is cyclic.
\end{example}
\begin{proof}
    Let $F$ be a field and $A$ a finite subgroup of $F^\times$.
    So $A$ is a finite abelian group, and if it is not cyclic, then by Theorem \ref{fin_abe_struct}, it contains a subgroup isomorphic to $C_m\times C_m$ for some $m\ge 2$, but then $f(X)=X^m-1$ has at least $m^2$ roots, contradicting the preceding lemma.
\end{proof}
\begin{proposition}
    Any finite integral domain is a field.
\end{proposition}
\begin{proof}
    Consider a finite integral domain $R$ and $0\neq a\in\mathbb R$.
    The map $\phi:R\to R$ by $r\mapsto ra$.
    This map is injective since $R$ is an integral domain, but then it is automatically surjective since $R$ is finite.
    So there is some $r$ such that $ra=1$.
\end{proof}
Combining these two gives that every finite integral domain has cyclic multiplicative group.
\begin{theorem}
    Let $R$ be an integral domain, then there is a field $F$ with the following properties:\\
    1. $R\le F$.\\
    2. Every element of $F$ can be written as $ab^{-1}$ where $a,b\in R$.
\end{theorem}
Consequently, such an $F$ is the unique minimal field containing $R$.
$F$ is called the field of fractions.
\begin{example}
    The field of fractions of $\mathbb Z$ is $\mathbb Q$.
\end{example}
\begin{proof}
    Consider the set $F=(R\times R\setminus\{0\})/\sim$ where
    $$(a,b)\sim (c,d)\iff ad=bc$$
    We write the equivalence class containing $(a,b)$ as $a/b$.
    One can show that this is an equivalence relation since $R$ is an integral domain and that the following operations are well-defined:
    $$(a/b)+(c/d)=(ad+bc)/(bd),(a/b)(c/d)=(ac)/(bd)$$
    $F$ is obviously a field under these two operations.
    Also we can embed $R$ into $F$ by $r\mapsto r/1$ and we have $a/b=(a/1)(1/b)=(a/1)(b/1)^{-1}=ab^{-1}$, so this is the field we want.
\end{proof}
\begin{example}
    1. The field of fraction of the Gaussian integers $\mathbb Z[i]$ is the set $\{ab^{-1}:a,b\in\mathbb Z[i]\le\mathbb C\}$.
    In fact, $F$ is exactly numbers in the form $p+iq,p,q\in\mathbb Q$.\\
    2. The field of fraction of the polynomial ring $R[X]$ over a ring $R$ is called the field of fractions $R(X)$ of $R$.
\end{example}
\subsection{Prime and Maximal Ideals}
\begin{lemma}
    A ring $R$ is a field iff its only ideas are $\{0\},R$.
\end{lemma}
\begin{proof}
    Trivial.
\end{proof}
\begin{definition}
    Let $S$ be a collection of subsets of a set $X$.
    $A\in S$ is maximal if there does not exists $B\in S$ such that $A\subsetneq B$.\\
    An ideal $I\unlhd R$ is maximal if it is maximal in the set of all proper ideas $\mathcal I_R=\{J\unlhd R:\{0\}\subsetneq J\subsetneq R\}$.
\end{definition}
\begin{proposition}
    Let $I\unlhd R$, then $R/I$ is a field iff $I$ is maximal.
\end{proposition}
\begin{proof}
    $R/I$ is a field iff $I/I,R/I$ are the only ideals of $R/I$, which happens iff $I$ and $R$ are the only ideals of $R$ containing $I$ iff $I$ is maximal.
\end{proof}
\begin{definition}
    An ideal $I\unlhd R$ is prime if $I\neq R$ and $ab\in I$ implies that at least one of $a,b$ is in $I$.
\end{definition}
\begin{example}
    The prime ideals of $\mathbb Z$ are $p\mathbb Z$ with $p$ prime or $0$.
    Incidentally (or not), $p\mathbb Z$ are also all maximal ideals of $\mathbb Z$.
\end{example}
\begin{proposition}
    Let $I\unlhd R$, then $I$ is prime iff $R/I$ is an integral domain.
\end{proposition}
\begin{proof}
    $I$ is prime iff $ab\in I\implies a\in I\lor b\in I$ iff $ab+I=I\implies a+I=I\lor b+I=I$ iff $I$ is an integral domain.
\end{proof}
\begin{remark}
    Combining the results reveals that every maximal ideal is prime.
\end{remark}
\begin{remark}
    If $\operatorname{char}(R)=n\ge 2$, then $\mathbb Z/n\mathbb Z\le R$, hence $n$ is prime.
    In particular, the characteristic of a field $F$ is either $0$ or a prime number.
    When the field has characteristic $0$, then $\mathbb Z\le F$, hence $\mathbb Q\le F$ since $F$ is a field.
\end{remark}