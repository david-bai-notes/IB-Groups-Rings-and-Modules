\section{Rings}
\begin{definition}
    A ring is a triple $(R,+,\cdot)$ consisting of a set $R$ and two binary operations $+,\cdot:R\times R\to R$, called addition and multiplication, such that\\
    1. $(R,+)$ is an abelian group.
    Its identity is called $0=0_R$.\\
    2. $\cdot$ is associative and has an identity called $1=1_R$.\\
    3. $\forall x,y,z\in R,x\cdot(y+z)=x\cdot y+x\cdot z,(x+y)\cdot z=x\cdot z+y\cdot z$.
\end{definition}
\begin{remark}
    1. Again closure is not explicitly listed as an axiom, but it is included in the well-definedness of $+,\cdot$, so to verify something is a ring, we still have to check the closure.\\
    2. For $x\in R$, we write $-x$ for its additive inverse and abbreviate $x+(-y)=x-y$.\\
    3. $0=0\cdot x-0\cdot x=(0+0)\cdot x-0\cdot x=0\cdot x+0\cdot x-0\cdot x=0\cdot x$.
    Similarly $x\cdot 0=0$.\\
    4. $-x=0-x=0\cdot x-x=(1+(-1))\cdot x-x=x+(-1)\cdot x-x=(-1)\cdot x$.\\
    5. Using 4 and other axioms, we can deduce back the property that addition is commutative.
\end{remark}
\begin{definition}
    If $\cdot$ is commutative as well, we say $R$ is a commutative ring.
\end{definition}
In this course, we are only interested in commutative rings.
When we say ``ring'' afterwards, we will always imply commutativity.
\begin{definition}
    Given a ring $R$, a subring $S$ is a subset of $R$ such that $S$ is a ring under the same addition and multiplication restricted to $S\times S$ and the same identity elements.
\end{definition}
\begin{example}
    1. $\mathbb Z$ is a ring under the familiar operations.\\
    2. The Gaussian integers $\mathbb Z[i]=\{a+ib:a,b\in\mathbb Z\}$ is a ring and is a subring of $\mathbb C$.\\
    3. The set $\mathbb Q[\sqrt{2}]=\{p+q\sqrt 2:p,q\in\mathbb Q\}$.\\
    4. $\mathbb Z[1/p]=\{m/p^n:m\in\mathbb Z,n\in\mathbb N\}$ where $p$ is a prime gives a ring that is a subring of $\mathbb Q$.\\
    5. $\mathbb Z/n\mathbb Z$ is a ring for any natural number $n$.
\end{example}
We can also construct new rings from old.
\begin{definition}
    Given rings $R,S$, the product ring is the Cartesian product $R\times S$ with operations
    $$(r_1,s_1)+(r_2,s_2)=(r_1+r_2,s_1+s_2),(r_1,s_1)\cdot(r_2,s_2)=(r_1\cdot r_2,s_1\cdot s_2)$$
    So $0_{R\times S}=(0_R,0_S)$ and $1_{R\times S}=(1_R,1_S)$.
\end{definition}
\begin{definition}
    Let $R$ be a ring and $X$ a set, then the set of functions $X\to R$ is a ring under pointwise operations.
    So $(f+g)(x)=f(x)+g(x),(f\cdot g)(x)=f(x)\cdot g(x)$.
\end{definition}
Further interesting examples appear as subrings of it.
For example, the set of all continuous functions $\mathbb R\to\mathbb R$ is a ring.
\begin{definition}
    Let $R$ be a ring and $S$ the set of all sequences of elements $r_0,r_1,r_2,\ldots$ of $R$ that is eventually zero.
    Consider the operations
    $$(a_0,a_1,\ldots)+(b_0,b_1,\ldots)=(a_0+b_0,a_1+b_1,\ldots)$$
    and
    $$(a_0,a_1,\ldots)\cdot (b_0,b_1,\ldots)=(c_1,c_2,\ldots),c_n=\sum_{i=0}^na_ib_{n-i}$$
\end{definition}
One can verify that this is indeed a ring.
We can identify $R$ as a subring of $S$ by $r\mapsto (r,0,0,\ldots)$.
Define $X=(0,1,0,0,\ldots)$, so $X^n=(0,\ldots,0,1,0,0,\ldots)$ where the $1$ occurs as the $n^{th}$ entry (counting from $0$).
So $S$ is generated by $X$ and $R$, hence every element of $S$ can be identifies as a sum
$$\sum_{i=0}^na_iX^i,a_i\in\mathbb R$$
So this ring $S=R[X]$ is called the polynomial ring over $R$.
We define the degree of a polynomial to be the largest $i$ such that the coefficient $a_i$ is nonzero.\\
We can obviously identify each polynomial $a_0+a_1X+\cdots +a_nX^n$ as a function $x\mapsto a_0+a_1x+\cdots +a_nx^n$, however
\begin{remark}
    Let $R=\mathbb Z/p\mathbb Z$ where $p$ is a prime, and $f(X)=X^p-X$, so we can identify a function $R\to R$ by $x\mapsto x^p-x=0$.
\end{remark}
\begin{definition}
    For a ring $R$, we can define multivariate polynomial ring $R[X_1,\ldots,X_n]=R[X_1,\ldots,X_{n-1}][X_n]$ inductively.
\end{definition}
\begin{definition}
    Given a ring, we can define the power series ring $R[[X]]$ that is all power series in $X$, which are formal sequence of coefficient following the same operation as polynomial rings.
\end{definition}
\begin{definition}
    The Laurent polynomials $R[X,X^{-1}]$ is defined as functions from $\mathbb Z\to R$ taking finitely many nonzero values following yet again the same operation.
    We write the elements in the form $\cdots +a_{-2}X^{-2}+a_{-1}X^{-1}+a_0+a_1X+a_2X^2+\cdots$ where only finitely many $a_i$ is nonzero.
\end{definition}
\begin{definition}
    Given a ring $R$, an element $r\in R$ is called a unit if it has a multiplicative inverse.
    We write $r^{-1}$ for that inverse.
\end{definition}
A warning is that whether or not an element is a unit depends on the ring, for example $2$ is a unit in $\mathbb Q$ but not in $\mathbb Z$.\\
The set of all inverse in a ring forms a group $R^\times$ under multiplication.
So $\mathbb Z^\times=\{\pm 1\}$ and $\mathbb Q^\times=\mathbb Q\setminus\{0\}$.
Sometimes we write $R^\times$ as $R^\star$.
\begin{definition}
    A field is a ring with $0\neq 1$ and that every nonzero element is a unit.
\end{definition}
\begin{example}
    $\mathbb Q,\mathbb R,\mathbb C,\mathbb Z/p\mathbb Z$ ($p$ prime) are fields.
\end{example}
There is obviously a reason why we need $0\neq 1$.
\begin{remark}
    If $R$ is a ring such that $0=1$, then $\forall x\in R,x=1\cdot x=0\cdot x=0$, so every element of this ring is zero.
\end{remark}
\begin{lemma}
    Let $f,g\in R[X]$ be polynomials, and suppose that $g$ is nonconstant and that the leading coefficient of $g$ is a unit, then there exists a quotient with remainder.
    That is, $\exists q,r\in R[X]$ such that $f=qg+r$ with $\deg r<\deg g$.
\end{lemma}
\begin{proof}
    Simple induction.
\end{proof}